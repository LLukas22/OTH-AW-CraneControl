\chapter{Übertragung}
In den Anwendungen ist ein TCP/IP Server implementiert, von dem der Krahn die Daten abgreifen kann. Dieser läuft auf Port 54000 und sendet die Daten
als 5 Byte großes Array.\\
\\\
Diese Bytes teilen sie wie folg auf:
\begin{table}[H]
    \centering
    \begin{tabular}{r|c|c|c|c|c}
        Byteposition & 0     & 1       & 2     & 3      & 4     \\\hline
        Bedeutung    & Power & Ein/Aus & Links & Rechts & Auf	Ab \\\hline
        Werte        & 0,1   & 0…100   & 0.100 & 0,1,2  & 0,1,2
    \end{tabular}
    \caption{Übertragungsprotokoll}
    \label{tab:Protokoll}
\end{table}
Zum Testen wurde ein Dummy Client in Python geschrieben, der dauerhaft Daten anfordert und diese in der Konsole ausgiebt.
\begin{figure}[H]
    \centering
    \includegraphics[width=0.6\textwidth]{PythonClient.png}
    \caption[Python Client]{Python Client}
    \label{fig:Python Client}
\end{figure}