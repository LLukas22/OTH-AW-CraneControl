\chapter{Vergleich}
\section{Vorteile}
\textbf{Kinect}
\begin{itemize}
    \item Einfache Implementierung, da ein Großteil der Datenverarbeitung direkt in der Kinect stattfindet
    \item leicht erweiterbar ohne neue Datensätze zu trainieren
    \item unabhängig von der PC-Leistung
    \item Theoretisch kann man ich Nachhinein noch über TF die Bilder weiter auswerten
\end{itemize}
\textbf{Tensorflow}
\begin{itemize}
    \item Keine spezielle Kamera notwendig.
    \item Theoretisch höhere Genauigkeit.
    \item Kontrolle über den gesamten Prozess. (Kamera bis Model Output)
\end{itemize}
\newpage
\section{Nachteile}
\textbf{Kinect}
\begin{itemize}
    \item Es ist immer eine Kinect als Kamera notwendig
    \item Kinect Library nicht für jede Programmiersprache verfügbar
    \item Kinect ist am Aussterben
\end{itemize}
\textbf{Tensorflow}
\begin{itemize}
    \item Aufwendiges trainierten eines TF-Modells
    \item Hohe Genauigkeiten nur mit Großen Modellen möglich (Exponentieller Rechenleistungs zuwachs)
    \item Für Echtzeit-Erkennung wird eine GPU oder TPU benötigt.
\end{itemize}
\section{Fazit}
Beide Methoden haben einige Vor- \& Nachteile.\\
Es zeigt sich, das die Kinect eine einfachere Implementation bietet, aber die Variante mit Tensorflow vielseitiger einsetzbar ist.