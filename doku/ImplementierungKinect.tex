\chapter{Implementierung}
\section{Kinect - C\#}
\subsection{Die Kinect}
Die Kinect ist ein vielseitiges Stück Hardware, das eigentlich für die xBox 360 als gegenstück zu Wii dienen sollte. Jedoch hat sie nie großen Erfolg gehabt.\\
Heutzutage wird sie va. nur noch für Kamervision-Projekte verwendet. Für diese bietet die Kinect viele Tools, um eine einfache Implementierung und Auswertung
zu ermöglichen.
\subsubsection{Kameras}
Die Kinect kann 3 verschiedene Kamerafeeds ausgeben. Darunter sind ein normales Farbiges Bild, eine Tiefenmapp und ein Infrarot Bild.\\
\begin{figure}[H]
    \centering
    \subfigure[Infrarot-Bild]{\includegraphics[width=0.4\textwidth]{Kinect IR picture.png}}
    \subfigure[Tiefen-Bild]{\includegraphics[width=0.4\textwidth]{Kinect depth picture.jpg}}
    \caption{Kamerafeeds der Kinect}
    \label{fig:Kamerafeeds}
\end{figure}
\subsubsection{Skelett}
Die Kinect kann Menschen erkennen, und ihnen Skelette zuordnen, um einfache Implementierung von gestenbasierten Anwendungen zu ermöglichen.
\begin{figure}[H]
    \centering
    \subfigure[Skelettpunkte]{\includegraphics[width=0.59\textwidth]{Kinect Joint Map.png}}
    \subfigure[Beispielbild]{\includegraphics[width=0.39\textwidth]{Kinect Skeletton picture.png}}
    \caption[Kinect Skelett]{Skelettpunkte des Körpers}
    \label{fig:Koerperpunkte}
\end{figure}
\subsubsection{Coordinate Mapping}
Durch überlagerung der Depthmap, mit dem Skelett kann man eine Maske für das Farbbild erstellen, und so einen Greenscreen Simulieren.
Ein Beispiel ist in \cite{KinectBackgroundRemoval} zu finden.\\
\subsubsection{Facedetection}
Weiterhin ist in der Kinect bereits eine vielzahl an Tools zu Gesichtserkennung integriert\cite{KinectFaceTracking}, es werden direkt die Position von Gesichtern und eine Facemap zurück gegeben.
Dadurch lässt sich sich z.B. auch für Windows Hello verwenden.\\\
\begin{figure}[H]
    \centering
    \includegraphics[width=0.7\textwidth]{kinect-face-hd-2.png}
    \caption[Kinect Facemap]{Kinect Facemap}
    \label{fig:Kinect Facemap}
\end{figure}
\subsubsection{3D-Bild}
Aufgrund der Tiefendaten und des Farbbildes lassen sich mit der Kinect auch 3D-Modelle erstellen.
Dazu giebt es direct von Microsoft die App 3D Scan\cite{3DScan}
\begin{figure}[H]
    \centering
    \includegraphics[width=0.7\textwidth]{Kinect-Fusion.png}
    \caption[Kinect Fusion]{Kinect Fusion}
    \label{fig:Kinect Fusion}
\end{figure}

\subsubsection{Gesten}
Die Kinect unterstützt die Erkennung von Handgesten. Vorgefertigt sind die Gesten Lasso, Offen und Geschlossen.
Weitere Gesten können über den Gesturebuilder, der mit dem KinectSDK\cite{KinectSDK} installiert wird, definiert werden.
Genaueres dazu wird im nächsten Kapitel erläutert.
\newpage
\subsection{Konzept}

Um eine Richtungsangabe über die Kinect zu bekommen, wird ein Skelett gesucht und von diesem die Hände verfolgt.
Die Hände werden dann überwacht ob eine Geste erkannt wird. Weiterhin wird das Tiefenbild in zusammenhang mit den Koordinaten der Hände verwendet, um im Bild, in der Benutzeroberfläche, die Erkannten Hände einzuzeichnen.\\
Auf der Basis der erkannten Gesten der Kinect und der Richtung der Lasso Geste soll eine Bewegungsanweisung an den Krahn gesendet werden.\\
Die Gesten sind wie folgt:
\begin{figure}[H]
    \centering
    \subfigure[Offen]{\includegraphics[width=0.32\textwidth]{Geste_offen.jpg}}
    \subfigure[Geschlossen]{\includegraphics[width=0.32\textwidth]{Geste_geschlossen.jpg}}
    \subfigure[Lasso]{\includegraphics[width=0.32\textwidth]{Geste_lasso.jpg}}
    \caption[Kinect Gesten]{Kinect Gesten. Bildquelle: eigene Bilder}
    \label{fig:Kinect Gesten}
\end{figure}

Die offene Hand wird nicht als Geste verwendet, da auch eine Möglichkeit bestehen muss kein Befehl zu senden. Mit dem schließen der linken oder rechten
Hand kann der Regler des Krahns oder die Betriebsmodi des Krahs umgeschalten werden. Über die Lasso Geste soll die Bewegung gesteuert werden, dazu
werden über das Skelett des zugehörigen Körpers die Koordinaten von Handmittelpunkt und Fingerspitze verglichen.


\subsection{Code}

\subsubsection{Krahnsteuerung}
Im Hauptprogramm wird das Winform Fenster erzeugt und auch der Server verwaltet. Der Server läuft in einem eingenen Task und ist somit weitestgehend
unabhängig vom rest des Programms und antwortet dem Krahn auf anfrage mit den Bewegungsdaten.

\subsubsection{Kinect}
Hier wird die Kinect initialisiert und verwaltet. Ausgegeben werden die Bilder für die anzeige in der Windowsoberfläche und ein Array mit den erkannten
Befehlen. Damit wird der Speicher im Server aktualisiert, vom der der Server ließt um dem Krahn daten zu senden.\\
Die erkennung der Gesten für den Regler und die Betriebsmodi ist durch die Kinect bereits zu einem grißem Teil implementiert. Man muss sich nur
die aktullen Gesten mit der zugehörigen Hand ausgeben und bei der Kombination Rechte Hand ist geschlossen wird der Regler umgeschalten bzw. Linke Hand
ist geschlossen dann wird der Betriebsmodus umgeschalten.\\
Die Richtungsgesten gestalten sich etwach komplizeriter, da zur Geste "Lasso" auch noch eine Richtung benötigt wird. Dazu kann das Skelett, welches die
Kinect für jeden Körper im Bild erstellt, verwendet werden. Mann sieht sich die Punkte für Handmittelpunkt und Fingerspitze der Hand an, bei der
das "Lasso" erkannt wurde an und vergleicht den Unterschied der x und y Koordinaten.
\\\
\lstinputlisting[label={lst:Bewegungsanweisung},
    numbers={none},
    caption={Berechnung der Bewegungsrichtung}]
{CodeSamples/HandGestureDirections.cs}

\newpage
Bildlich dargestellt ist der Handmittelpunkt bei (0,0). Abhängig davon wo die Fingerspitzen in Realtion dazu sind werden die roten Bereiche
herrausgefiltert und ansonsten die jeweilige Richtung zurück gegeben.

\begin{figure}[H]
    \centering
    \begin{tikzpicture}
        \newcommand{\xa}{5};
        \newcommand{\xb}{3};
        \draw[draw=lightgray, step=1cm] (-{\xa}, -{\xa}) grid ({\xa}, {\xa});
        \draw[thick, -latex] (-{\xa+0.5}, 0) -- ({\xa-0.5}, 0) node[below] {$x$};
        \draw[thick, -latex] (0, -{\xa+0.5}) -- (0, {\xa-0.5}) node[right] {$y$};
        \node [draw] at (0,{\xa+0.5}) {UP};
        \node [draw] at (0,-{\xa-0.5}) {DOWN};
        \node [draw] at (-{\xa-1},0) {LEFT};
        \node [draw] at ({\xa+1},0) {RIGHT};
        \draw [draw=red, fill=red] (0,0) -- ({\xb},{\xa}) -- ({\xa},{\xa}) -- ({\xa},{\xb}) -- cycle;
        \draw [draw=red, fill=red] (0,0) -- (-{\xb},-{\xa}) -- (-{\xa},-{\xa}) -- (-{\xa},-{\xb}) -- cycle;
        \draw [draw=red, fill=red] (0,0) -- ({\xb},-{\xa}) -- ({\xa},-{\xa}) -- ({\xa},-{\xb}) -- cycle;
        \draw [draw=red, fill=red] (0,0) -- (-{\xb},{\xa}) -- (-{\xa},{\xa}) -- (-{\xa},{\xb}) -- cycle;
    \end{tikzpicture}
    \caption{Filter der Richtungsgesten.}
    \label{fig:Richtungsfilter}
\end{figure}

\subsubsection{PubSubEvent}
Modul, das die Datenweitergabe an das Hauptprogramm bzw. den Server Regelt und dafür eine einheitliche Schnitstelle definiert. Dadurch können
theoretisch sehr einfach weiter Module hinzugefügt werden.
\\\
\lstinputlisting[label={lst:Bewegungsanweisungsliste},
    numbers={none},
    caption={Übertragung der Erkannten Daten}]
{CodeSamples/Instructions.cs}

\newpage
\subsection{User Interface}
Das UI ist simple gehalten, um eine einfache Anwendung zu ermöglichen. Unten links wird der erkannte Befehl angezeigt. Rechts kann über das
Zahnradsymbol ein Optionspannel ausgeklappt werden. Dort kann der Port, auf dem sich der Krahn verbinnden soll angegeben werden(Standard ist 54000),
darunter ist zu erkennen, ob ein Client verbunden ist. Weiterhin kann eingestellt werden, ob das Farb-, Infrarot- oder Tiefenbild angezeigt werden soll.
Außerdem kann die Beschläunigung des Krahns verändert werden. Über Knöpfe kann der Server pausiert werden, so das keine Pakete mehr gesendet werden,
oder auch komplett ausgeschalten werden zum Neustarten.
\begin{figure}[H]
    \centering
    \includegraphics[width=0.7\linewidth]{Kinect UI.png}
    \caption[Kinect UI]{Kinect UI}
    \label{fig:Kinect UI}
\end{figure}