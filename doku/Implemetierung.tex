\chapter{Implementierung}


\section{Kinect - C\#}
\subsection{Konzept}
Kinect erkennt 3 Gesten + NotTracked + Unknown

\begin{figure}[H]
    \centering
    \subfigure[Offen]{\includegraphics[width=0.32\textwidth]{Geste_offen.jpg}}
    \subfigure[Geschlossen]{\includegraphics[width=0.32\textwidth]{Geste_geschlossen.jpg}}
    \subfigure[Lasso]{\includegraphics[width=0.32\textwidth]{Geste_lasso.jpg}}
    \caption[Kinect Gesten]{Kinect Gesten. Bildquelle: eigene Bilder}
	\label{fig:Kinect Gesten}
\end{figure}


\subsection{Code}
\subsubsection{Gestenerkennung}
\begin{figure}[H]
    \centering
    \begin{tikzpicture}
        \newcommand{\xa}{5};
        \newcommand{\xb}{3};
        \draw[draw=lightgray, step=1cm] (-{\xa}, -{\xa}) grid ({\xa}, {\xa});
        \draw[thick, -latex] (-{\xa+0.5}, 0) -- ({\xa-0.5}, 0) node[below] {$x$};
        \draw[thick, -latex] (0, -{\xa+0.5}) -- (0, {\xa-0.5}) node[right] {$y$};
        \node [draw] at (0,{\xa+0.5}) {UP};
        \node [draw] at (0,-{\xa-0.5}) {DOWN};
        \node [draw] at (-{\xa-1},0) {LEFT};
        \node [draw] at ({\xa+1},0) {RIGHT};
        \draw [draw=red, fill=red] (0,0) -- ({\xb},{\xa}) -- ({\xa},{\xa}) -- ({\xa},{\xb}) -- cycle;
        \draw [draw=red, fill=red] (0,0) -- (-{\xb},-{\xa}) -- (-{\xa},-{\xa}) -- (-{\xa},-{\xb}) -- cycle;
        \draw [draw=red, fill=red] (0,0) -- ({\xb},-{\xa}) -- ({\xa},-{\xa}) -- ({\xa},-{\xb}) -- cycle;
        \draw [draw=red, fill=red] (0,0) -- (-{\xb},{\xa}) -- (-{\xa},{\xa}) -- (-{\xa},{\xb}) -- cycle;
    \end{tikzpicture}
    \caption{Filter der Richtungsgesten.}
    \label{fig:Richtungsfilter}
\end{figure}
\subsubsection{Kinect}
Übernimmt die Initialiserung und Verabeitung der Kinect. Ausgegeben werden eine Bitmap des Bildes und in einem Array die Richtungsanweisungen.
Im Bild werden die erkannten Hände umrandet.

\subsubsection{Krahnsteuerung}
Ist das Hauptprogramm in dem das Winform und der Server gehandhabt werden.

\subsubsection{SharedRessources}
Klasse, in der die Übergabe an das Hauptprogramm geregelt wird, so dass dieses um andere Eingabemöglichkeiten erweitert werden könnte.

\subsection{UI}
UI
\begin{figure}[H]
    \centering
        \includegraphics[width=0.7\linewidth]{Kinect UI.png}
        \caption[Kinect UI]{Kinect UI}
        \label{fig:Kinect UI}
\end{figure}




\newpage
\section{Tensorflow - Python}

\textbf{TensorflowTrainer:} \\
    Git: \url{https://git-scm.com/downloads}\\
    Suppoerted Models: \url{https://github.com/tensorflow/models/blob/master/research/object_detection/g3doc/tf2_detection_zoo.md}\\
    Microsoft C++ Build Tools: \url{https://visualstudio.microsoft.com/thank-you-downloading-visual-studio/?sku=BuildTools&rel=16}\\
    Für GPU-Support: \url{https://www.tensorflow.org/install/gpu}\\


\textbf{Für PI:} \\
    TFLight : \url{https://www.tensorflow.org/lite/guide/python}\\
    Apt.Get : sudo apt-get install python3-pil python3-pil.imagetk libatlas-base-dev libhdf5-dev libhdf5-serial-dev libatlas-base-dev libjasper-dev libqtgui4 libqt4-test\\
    Pip3: pip3 install Pillow,opencv-python\\


