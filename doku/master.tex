\documentclass[12pt,oneside]{report}
\usepackage[T1]{fontenc}		% Einstellungen fuer Umlaute usw.
\usepackage[utf8x]{inputenc}
\usepackage[ngerman]{babel}

\catcode`\_=12 % change catcode of "_" to "other" (no. 12)
\catcode`\"=12 % change catcode of """ to "other" (no. 12)

\usepackage{parskip}			% Einstellungen fuer Absaetze: Abstand statt Einrueckung

\usepackage[a4paper,			% Papierformat A4
	    left=2.5cm,				% linker Rand
	    right=2.5cm,			% rechter Rand
	    top=1cm,				% oberer Rand
	    bottom=1cm,			% unter Rand
	    marginparsep=5mm,		% Abstand der Randnotizen
	    marginparwidth=10mm, 	% Breite der Randnotizen
	    headheight=7mm,			% Hoehe der Kopfzeile
	    headsep=1.2cm,			% Abstand der Kopfzeile
	    footskip=1.5cm,			% Abstand der Fusszeile
	    includeheadfoot]{geometry}

        \usepackage{fancyhdr}						% Konfiguration von Kopf- und Fusszeilen
        \pagestyle{fancy}							% Seitenstil 'fancy'
        \fancyhf{}									% vorhandene Einstellungen loeschen
        \setlength{\headwidth}{\textwidth}			% Kopf- und Fusszeile so breit wie der Haupttext
        \fancyfoot[R]{\thepage} 					% Festlegung des Seitenstils: Seitenzahlen in der Fusszeile rechts
        \fancyfoot[L]{\leftmark}					% Kapitelnr. und -Bezeichnung in der Fusszeile links
        \fancyhead[R]{\IhreArbeit}					% "Bachelorarbeit" in der Kopfzeile rechts
        \fancyhead[L]{\ErsterName\ , \ZweiterName}	% Vorname und Name in der Kopfzeile links
        \renewcommand{\chaptermark}[1]{			% Definition der Ausgabe des Kapitels
          \markboth{Kapitel \thechapter. #1}{}}
        \renewcommand{\headrulewidth}{0.5pt}		% Trennlinie zwischen Kopfzeile und Haupttext
        \renewcommand{\footrulewidth}{0.5pt}		% Trennlinie zwischen Haupttext und Fusszeile
        \fancypagestyle{plain}{					% Anpassung des Seitenstils 'plain' bei Beginn neuer Kapitel
          \fancyhf{}								% Vorbelegung loeschen
          \fancyfoot[C]{\thepage}					% Seitenzeilen in der Fusszeile mittig
          \fancyhead[R]{\IhreArbeit}				% "Bachelorarbeit" in der Kopfzeile rechts
          \fancyhead[L]{\ErsterName\ , \ZweiterName}	% Vorname und Name in der Kopfzeile links
        }
        
        %\usepackage{amsmath}			% Pakete fuer den Mathematikmodus
        %\usepackage{amssymb}
        \usepackage[intlimits]{empheq}
        \usepackage{subfigure}
        \usepackage{float}
        
        \usepackage[sc]{mathpazo}		% Schriftart Palatino fuer Haupttext und Mathematikmodus
        \usepackage{pifont}				% zusaetzliche Symbole
        
        \usepackage[format=hang,		% Einstellung fuer Bildunterschriften
                    font={footnotesize},
                    labelfont={bf},
                    margin=1cm,
                    aboveskip=5pt,
                    position=bottom]{caption}
        
        \usepackage{graphicx}	   % Einbinden von Grafiken (jpg, png, pdf, ...)
        \graphicspath{{images/}}   % Suchpfad für Grafikdateien
        
        \usepackage[svgnames,table,hyperref]{xcolor} 	% Verwendung von Farben
        \usepackage{tikz}								% Erstellen von Grafiken
        \usetikzlibrary{positioning,arrows,plotmarks} % TikZ-Bibliotheken
        %\usepackage{pgfplots}                           % Darstellung von Plots, Funktionen, Graphen usw.
        
        %
        % Weitere Pakete
        %
        \usepackage{listings}			% Darstellung von Quellcode
        \usepackage{xcolor}
        \usepackage{bera}

        %Default Code Style
        \definecolor{codegreen}{rgb}{0,0.6,0}
        \definecolor{codegray}{rgb}{0.5,0.5,0.5}
        \definecolor{codepurple}{rgb}{0.58,0,0.82}
        \definecolor{backcolour}{rgb}{0.95,0.95,0.92}

        \colorlet{punct}{red!60!black}
        \definecolor{background}{HTML}{EEEEEE}
        \definecolor{delim}{RGB}{20,105,176}
        \colorlet{numb}{magenta!60!black}

        \lstdefinestyle{mystyle}{
            backgroundcolor=\color{backcolour},   
            commentstyle=\color{codegreen},
            keywordstyle=\color{magenta},
            numberstyle=\tiny\color{codegray},
            stringstyle=\color{codepurple},
            basicstyle=\ttfamily\footnotesize,
            breakatwhitespace=false,         
            breaklines=true,                 
            captionpos=b,                    
            keepspaces=true,                 
            numbers=left,                    
            numbersep=5pt,                  
            showspaces=false,                
            showstringspaces=false,
            showtabs=false,                  
            tabsize=2
        }

        \lstdefinelanguage{json}{
        basicstyle=\normalfont\ttfamily,
        numbers=left,
        numberstyle=\scriptsize,
        stepnumber=1,
        numbersep=8pt,
        showstringspaces=false,
        breaklines=true,
        frame=lines,
        backgroundcolor=\color{background},
        literate=
        *{:}{{{\color{punct}{:}}}}{1}
        {,}{{{\color{punct}{,}}}}{1}
        {\{}{{{\color{delim}{\{}}}}{1}
        {\}}{{{\color{delim}{\}}}}}{1}
        {[}{{{\color{delim}{[}}}}{1}
        {]}{{{\color{delim}{]}}}}{1},
}

        \lstset{language=Python, basicstyle=\ttfamily, numbers=left,style=mystyle}
        %
        \usepackage[square,numbers,sort]{natbib} % Referenzen
        %
        %\usepackage[european, siunitx]{circuitikz}	% Darstellung von Schaltungen
        %
        %\usepackage{enumerate}			% Formatierung nummerierter Listen
        
        \usepackage{microtype,relsize}					% Wird verwendet, um Nachnamen auf Titelseite gesperrt darzustellen
        \newcommand*{\Sperren}[1]{\textls*[100]{#1}}

        
% 
% Persoenliche Angaben
% 
\newcommand*{\ErsterName}{Lukas Kreussel}
\newcommand*{\ZweiterName}{Andreas Ziegler}
\newcommand*{\IhrStudiengang}{Elektro- und Informationstechnik}
\newcommand*{\IhreArbeit}{Projektarbeit}
\newcommand*{\IhrTitelDE}{Verbesserung der Gestensteuerung auf Basis eines Kinect 2.0 Moduls}
\newcommand*{\Professor}{Prof. Dr.-Ing. Armin Wolfram}
\newcommand*{\IhreSchluesselwoerter}{Kinect, Tensorflow, Gestensteuerung}


\usepackage[bookmarks, raiselinks, pageanchor, % PDF-Einstellungen
            hyperindex, colorlinks,
            citecolor=black, linkcolor=black,
            urlcolor=black, filecolor=black,
            menucolor=black]{hyperref}
\hypersetup{pdftitle={\IhrTitelDE},%
            pdfauthor={\ErsterName\ \ZweiterName},%
            pdfsubject={\IhreArbeit},%
            pdfkeywords={\IhreSchluesselwoerter}}



\begin{document}
\pagenumbering{roman}

\thispagestyle{empty}
\begin{center}
  \begin{figure}[H]
    \centering
    \includegraphics[width=0.7\linewidth]{OTH-Logo.jpg}
    \label{fig:OTH-Logo}
  \end{figure}
  \Large
  Ostbayerische Technische Hochschule Amberg-Weiden\\
  Fakultät Elektrotechnik, Medien und Informatik\\[1cm]
  Studiengang \IhrStudiengang\\[1cm]
  \textbf{\IhreArbeit}\\[1cm]
  von\\[1cm]
  \textbf{\ErsterName\ , \ZweiterName}\\[1cm]
  \textbf{\IhrTitelDE}\\[1cm]
\end{center}
\vspace*{3cm}

\vspace*{1cm}
\underbar{Professor:}\qquad\Professor
\clearpage
\tableofcontents
\newpage


\pagenumbering{arabic}
\chapter{Einleitung}


\section{Zielsetzung}
Mit der Kinect 2.0 oder eine Raspberry Pi mit Kamera sollen 2 Möglichkeiten zur Verbesserung der Gestensteuerung des Krans im Labor,
Abbildung 1.1, implementiert werden.
\begin{figure}[H]
	\centering
	\includegraphics[width=0.9\linewidth]{Krahn.jpg}
	\caption[Krahn]{Krahn. Bildquelle: Eigenes Bild}
	\label{fig:Krahn}
\end{figure}

\newpage
\section{Kinect}
\begin{figure}[H]
	\centering
	\includegraphics[width=0.7\linewidth]{Xbox-One-Kinect.jpg}
	\caption[Kinect v2]{Kinect v2. Bildquelle: Evan-Amos / Wikipedia public domain}
	\label{fig:pi4}
\end{figure}

Die Kinect 2 von Microsoft ist der Nachfolger der Ursprüglichen Kinect für die xBox 360 und wurde vorallem für Spiele entwockelt,
fand jedoch in diesem Gebiet wenig Markterfolg. Darum wurde die Kinect für xBox nach der Kinect 2 eingestellt. Seit dem findet die Kinect vor allem
Anwendung in Kameravison Anwendungen. Beispielsweise ist hier das Ganzkörpertracking für VR-Anwendungen zu nennen. Durch diese Entwicklung hat sich
Microsoft dazu entschieden, die Kinect Azure für Unternehmen zu entwickeln und als Entwicklungsumgebung zu vermarkten.\\
Durch die 3 Kameras, wovon eine Infrarot- und eine Tiefenkamera verbaut sind, und den on-Board Prozessor kann die Kinect in Anwendungen viel Arbeit
abnehmen, da sie bereits viele Daten auswertet und verarbeitet. So giebt die Kinect dem Anwender bis zu 6 Ganzkörper Skelette mit einzelnen "Gelenken"
die danach weiter verarbeitet werden können. Außerdem ist einen Gesten und Mimikerkennung bereits mit verbaut.\\
Für die Entwicklung von Kinectanwendungen stellt Microsoft ein umfangreiches Software Developer Kit SDK 2.0 \cite{KinectSDK} zu Verfügung. Darüber kann
ein Kinectobject erzeugt werden, dass alle Sensorausgaben und Skelettdaten als Membervariablen ausgiebt.

\newpage
\section{Tensorflow}
\begin{figure}[H]
	\centering
	\includegraphics[width=0.7\linewidth]{TensorFlow/tf_logo.png}
	\label{fig:TF Logo}
\end{figure}
TensorFlow ist ein, von Google entwickeltes, Framework zur datenstromorientierten Programmierung.
Populäre Anwendung findet TensorFlow im Bereich des maschinellen Lernens.
Seit 2020 wurde die TensorFlow Object Detection API offiziel in Tensorflow 2.0 integriert und ermöglicht das einfache und schnelle erstellen von ObjectDetection Modellen.
Hierfür werden von Google mehrere vortränierte Modelle über den Detection Model Zoo bereitgestellt, die innerhalb von wenigen Stunden und mit relativ kleinen Datenmengen auf einen neuen Datensatz gangepasst werden können.

\newpage
\section{Anforderungen}
Der Krahn soll sich über Wlan an einem Server, der lokal auf einem Rechner im Labor läuft, anmelden und dieser soll die Daten aus der Erkennung von
Kinect oder TensorFlow an den Krahn auf Anfrage übermitteln.\\
\\\
Für die Bewegung des Krahns sollen Gesten implementiert werden:
\begin{itemize}
	\item Bewegeung nach links und rechts des Krahns
	\item Einziehen und Ausfahren des Krahnseils
	\item Ein- und Ausschalten des Reglers für eine stabile Last
	\item Umschalten, zwischen den Betriebsmodi
\end{itemize}
\chapter{Herangehensweise}
Durch die Anforderung von 2 Möglichkeiten wurden diese Aufgeteilt und jeder hat sich mit einer Variante beschäftigt.

\section{Kinect}
Zuerst wurde das Kinect 2.0 SDK heruntergeladen und installiert. Danach wurde ein Beispielproject gesucht \cite{KinectFingerTracking}, welches sich
auch zum Teil im Fertigen Project wiederfindet. Es dient der Veranschaulichung, welche Hände momentan erkannt werden,
und zeichnet um diese einen Rahmen herum. Danach wurde auf Basis der Vordefinierten Gesten der Kinect Gesten für die Krahnbewegungen festgelegt und
die Geste für Bewgung mit einer Richtungsangabe über das Skelett versehen. Diese wurden dann an den Server gegebn, das dieser diese dem Krahn auf
Anfrage mitteilen kann.
\section{Tensorflow}
\chapter{Übertragung}
Das Programm öffnet einem Server mit dem es möglich ist die Daten per TCP/IP abzugreifen. Die Daten werden in ein 5 Byte großes Array geschrieben. Dies wird dann an den Client gesendet. Im Moment läuft der Server auf dem Port 54000. Es liegt ein Beispielprogramm eines Clients bei, welcher die Daten abgreift.


\begin{table}[H]
    \centering
    \begin{tabular}{r|c|c|c|c|c}
        Byteposition	&0&	1&	2&	3&	4\\\hline
        Bedeutung&	Power &Ein/Aus&	Links&	Rechts&	Auf	Ab\\\hline
        Werte	&0,1&	0…100&	0.100	&0,1,2&	0,1,2
    \end{tabular}
    \caption{Übertragungsprotokoll}
    \label{tab:Protokoll}
\end{table}


\chapter{Implementierung}
\section{Kinect - C\#}
\subsection{Konzept}
Auf der Basis der erkannten Gesten der Kinect und der Richtung der Lasso Geste soll eine Bewegungsanweisung an den Krahn gesendet werden.\\
Die Gesten sind wie folgt:
\begin{figure}[H]
    \centering
    \subfigure[Offen]{\includegraphics[width=0.32\textwidth]{Geste_offen.jpg}}
    \subfigure[Geschlossen]{\includegraphics[width=0.32\textwidth]{Geste_geschlossen.jpg}}
    \subfigure[Lasso]{\includegraphics[width=0.32\textwidth]{Geste_lasso.jpg}}
    \caption[Kinect Gesten]{Kinect Gesten. Bildquelle: eigene Bilder}
    \label{fig:Kinect Gesten}
\end{figure}

Die offene Hand wird nicht als Geste verwendet, da auch eine Möglichkeit bestehen muss kein Befehl zu senden. Mit dem schließen der linken oder rechten
Hand kann der Regler des Krahns oder die Betriebsmodi des Krahs umgeschalten werden. Über die Lasso Geste soll die Bewegung gesteuert werden, dazu
werden über das Skelett des zugehörigen Körpers die Koordinaten von Handmittelpunkt und Fingerspitze verglichen.

\subsection{Code}

\subsubsection{Krahnsteuerung}
Im Hauptprogramm wird das Winform Fenster erzeugt und auch der Server verwaltet. Der Server läuft in einem eingenen Task und ist somit weitestgehend
unabhängig vom rest des Programms und antwortet dem Krahn auf anfrage mit den Bewegungsdaten.

\subsubsection{Kinect}
Hier wird die Kinect initialisiert und verwaltet. Ausgegeben werden die Bilder für die anzeige in der Windowsoberfläche und ein Array mit den erkannten
Befehlen. Damit wird der Speicher im Server aktualisiert, vom der der Server ließt um dem Krahn daten zu senden.\\
Die erkennung der Gesten für den Regler und die Betriebsmodi ist durch die Kinect bereits zu einem grißem Teil implementiert. Man muss sich nur
die aktullen Gesten mit der zugehörigen Hand ausgeben und bei der Kombination Rechte Hand ist geschlossen wird der Regler umgeschalten bzw. Linke Hand
ist geschlossen dann wird der Betriebsmodus umgeschalten.\\
Die Richtungsgesten gestalten sich etwach komplizeriter, da zur Geste "Lasso" auch noch eine Richtung benötigt wird. Dazu kann das Skelett, welches die
Kinect für jeden Körper im Bild erstellt, verwendet werden. Mann sieht sich die Punkte für Handmittelpunkt und Fingerspitze der Hand an, bei der
das "Lasso" erkannt wurde an und vergleicht den Unterschied der x und y Koordinaten.
\\\
\lstinputlisting[label={lst:Bewegungsanweisung},
    numbers={none},
    caption={Berechnung der Bewegungsrichtung}]
{CodeSamples/HandGestureDirections.cs}

\newpage
Bildlich dargestellt ist der Handmittelpunkt bei (0,0). Abhängig davon wo die Fingerspitzen in Realtion dazu sind werden die roten Bereiche
herrausgefiltert und ansonsten die jeweilige Richtung zurück gegeben.

\begin{figure}[H]
    \centering
    \begin{tikzpicture}
        \newcommand{\xa}{5};
        \newcommand{\xb}{3};
        \draw[draw=lightgray, step=1cm] (-{\xa}, -{\xa}) grid ({\xa}, {\xa});
        \draw[thick, -latex] (-{\xa+0.5}, 0) -- ({\xa-0.5}, 0) node[below] {$x$};
        \draw[thick, -latex] (0, -{\xa+0.5}) -- (0, {\xa-0.5}) node[right] {$y$};
        \node [draw] at (0,{\xa+0.5}) {UP};
        \node [draw] at (0,-{\xa-0.5}) {DOWN};
        \node [draw] at (-{\xa-1},0) {LEFT};
        \node [draw] at ({\xa+1},0) {RIGHT};
        \draw [draw=red, fill=red] (0,0) -- ({\xb},{\xa}) -- ({\xa},{\xa}) -- ({\xa},{\xb}) -- cycle;
        \draw [draw=red, fill=red] (0,0) -- (-{\xb},-{\xa}) -- (-{\xa},-{\xa}) -- (-{\xa},-{\xb}) -- cycle;
        \draw [draw=red, fill=red] (0,0) -- ({\xb},-{\xa}) -- ({\xa},-{\xa}) -- ({\xa},-{\xb}) -- cycle;
        \draw [draw=red, fill=red] (0,0) -- (-{\xb},{\xa}) -- (-{\xa},{\xa}) -- (-{\xa},{\xb}) -- cycle;
    \end{tikzpicture}
    \caption{Filter der Richtungsgesten.}
    \label{fig:Richtungsfilter}
\end{figure}

\subsubsection{PubSubEvent}
Modul, das die Datenweitergabe an das Hauptprogramm bzw. den Server Regelt und dafür eine einheitliche Schnitstelle definiert. Dadurch können
theoretisch sehr einfach weiter Module hinzugefügt werden.
\\\
\lstinputlisting[label={lst:Bewegungsanweisung},
    numbers={none},
    caption={Übertragung der Erkannten Daten}]
{CodeSamples/Instructions.cs}

\newpage
\subsection{User Interface}
Das UI ist simple gehalten, um eine einfache Anwendung zu ermöglichen. Unten links wird der erkannte Befehl angezeigt. Rechts kann über das
Zahnradsymbol ein Optionspannel ausgeklappt werden. Dort kann der Port, auf dem sich der Krahn verbinnden soll angegeben werden(Standard ist 54000),
darunter ist zu erkennen, ob ein Client verbunden ist. Weiterhin kann eingestellt werden, ob das Farb-, Infrarot- oder Tiefenbild angezeigt werden soll.
Außerdem kann die Beschläunigung des Krahns verändert werden. Über Knöpfe kann der Server pausiert werden, so das keine Pakete mehr gesendet werden,
oder auch komplett ausgeschalten werden zum Neustarten.
\begin{figure}[H]
    \centering
    \includegraphics[width=0.7\linewidth]{Kinect UI.png}
    \caption[Kinect UI]{Kinect UI}
    \label{fig:Kinect UI}
\end{figure}
\section{Tensorflow - Python}
\subsection{Konzept}
Es wird ein Object-Detection-Model verwendet um über eine Webcam eine von sechs Gesten zu erkennen.
Die erkannte Geste wird in einem Buffer zwischengespeichert, und wenn mehrere Durchläufe das gleiche Ergebniss liefern wird ein Befehl an den Kran gesendet.

\begin{figure}[H]
    \centering
    \subfigure[Power]{\includegraphics[width=0.3\textwidth]{TensorFlow/Gestures/Power.PNG}}
    \subfigure[Up]{\includegraphics[width=0.3\textwidth]{TensorFlow/Gestures/Up.PNG}}
    \subfigure[Down]{\includegraphics[width=0.3\textwidth]{TensorFlow/Gestures/Down.PNG}}
    \subfigure[Left]{\includegraphics[width=0.3\textwidth]{TensorFlow/Gestures/Left.PNG}}
    \subfigure[Right]{\includegraphics[width=0.3\textwidth]{TensorFlow/Gestures/Right.PNG}}
    \subfigure[Toggle]{\includegraphics[width=0.3\textwidth]{TensorFlow/Gestures/Toggle.PNG}}
    \caption[Tensorflow Gesten]{Tensorflow Gesten. Bildquelle: eigene Bilder}
    \label{fig:Tensorflow Gesten}
\end{figure}
\newpage
\subsection{Tensorflow Trainer}
Um das trainieren der Modele zu erleichtern wurde eine C\# Application geschrieben, die diesen Vorgang weitgehend automatisiert.

\begin{figure}[H]
    \centering
    \subfigure{\includegraphics[width=0.9\textwidth]{TensorFlow/TensorflowTrainer.PNG}}
    \caption[Tensorflow Trainer]{Tensorflow Trainer}
    \label{fig:Tensorflow Trainer}
\end{figure}

\subsection{Trainieren eines eigene Models}

\subsubsection{Voraussetzungen:}

\begin{itemize}
    \item Git (\url{https://git-scm.com/downloads})
    \item C++ Build Tools (\url{https://visualstudio.microsoft.com/thank-you-downloading-visual-studio/?sku=BuildTools&rel=16})
    \item Python 3.7.X or newer (\url{https://www.python.org/downloads/})
    \item LabelImg (\url{https://github.com/tzutalin/labelImg})
    \item Nvidia Gpu (optional but highly recommended)
          \begin{itemize}
              \item CUDA Version (\url{https://www.tensorflow.org/install/gpu})
              \item cuDNN Version (\url{https://developer.nvidia.com/cudnn})
          \end{itemize}
\end{itemize}
\newpage
\subsubsection{Erstellen eines Datensatzes}
Um eine Datensatz zu erzeugen müssen zunächst mit LabelImg Boxen (XML PASCAL VOC Format) um die zu Erkennenden Objekte gezogen werden.
Hierbei sind 150-250 Bilder pro Objekt ein guter Richtwert.

\subsubsection{Python Packete}
Die benötigten Pip-Packete werden durch Klicken des \textit{Install}-Buttons im TensorflowTrainer installiert. Soll die GPU-Version von Tensorflow installiert werden muss die CheckBox  \textit{GPU Acceleration} abgehackt werden.
Der  \textit{Uninstall}-Button entfernt alle installierten Packete.

\subsubsection{Settings.json}
Nach erstem öffnen und schließen des TensorflowTrainer wird eine Settings.json erzeugt.
\lstinputlisting[language=json,label={lst:Settings.json},
    numbers={none},
    caption={Settings.json}]
{CodeSamples/Tensorflow/Settings.json}
Hier können Einstellungen angepasst werden, wobei das Feld \textbf{ModelUrl} das wichtigste ist, da hier das zu verwendende Model des TF-Model-Zoo (\url{https://github.com/tensorflow/models/blob/master/research/object_detection/g3doc/tf2_detection_zoo.md}) angegeben werden kann.

\subsubsection{Downloads und Setup}
Um die Restlichen Daten herunterzuladen, müssen die drei \textit{Download}-Buttons betätigt werden. Um die Downloads zu löschen kann der \textit{Clear Cache}-Button verwendet werden.
Nun muss in Environment ein Ordner augewähl werden, in dem das Training statfinden soll. Danach muss der \textit{Build}-Button betätigt werden. Der \textit{Open}-Button öffnet den ObjectDetection-Ordner.
Um die Installation abzuschließen muss zuerst der \textit{Compile *.protoc Files}- und danach der \textit{Execute setup.py}-Button betätigt werden. Nun werden alle ObjectDetection Module installiert und die Tensorflow Tests ausgeführt.
Wenn das Testergebnis \textit{OK} ist war die Installation erfolgreich.

\subsubsection{TF Records}
Als Nächstes müssen alle Bilder mit *.Xml Dateien nach ObjectDetection/images/input kopiert werden und die in LabelImg verwendeten Labels in richtiger Reihenfolgen in die Textbox eingetragen werden.
Danach die 4 Buttons von Links nach Rechts betätigen. Dies erzeugt test- und train.record im ObjectDetection-Ordner und autogeneriert die Labelmap und customPipeline in ObjectDetection/training.

\subsubsection{Pipeline Config}
In der customPipeline.config können nach bedarf DataAugemntation-Optionen hinzugefügt oder entfernt werden (\url{https://stackoverflow.com/questions/44906317/what-are-possible-values-for-data-augmentation-options-in-the-tensorflow-object}).
\lstinputlisting[language=json,label={lst:Pipeline},
    numbers={none},
    caption={Pipeline},
    firstline=137,
    lastline=153]
{CodeSamples/Tensorflow/customPipeline.config}

Die wichtigsten anpassbaren Werten sind:

\begin{itemize}
    \item \textbf{Batch_Size} was die auf einmal zu verarbeitenden Bilder sind. Falls nicht genug RAM im System verfügbar ist muss die Batch_Size gesenk werden.
    \item \textbf{learning_rate_base} Geht der Loss nach einigen Schritten gegen Unendlich ist die LearningRate zu hoch.
    \item \textbf{warmup_learning_rate} ähnlich wie learning_rate_base gilt aber nur für ersten warmup_steps.
\end{itemize}


\subsubsection{Training}
Das Training wird durch eine Click auf \textit{Start Training!} gestartet. Als andere Option kann man auch im ObjectDetection Ordner \textit{python model_main_tf2.py} ausführen. Tensorboard kann über einen Click auf \textit{Open Tensorboard} geöffnet werden.
Ändert der Loss sich nach einiger Zeit nicht mehr kann das Training abgebrochen werden und das fertige Model über einen Klick auf \textit{Export Graph} nach ObjectDetection/export/normal exportiert werden und über  \textit{Test on Webcam} getestet werden.
\\
\textbf{Anmerkungen}
\begin{itemize}
    \item[1.] Falls die Gpu nicht über genug Videospeicher verfügt besteht die Option Systemspeicher als Puffer zu verwednen. Dazu müssen in \textit{object_detection/model_main_tf2.py} nach dem Tensorflow import folgende Zeilen eingefügt werden:
          \lstinputlisting[language=python,label={lst:Gpu Growth},
              numbers={none},
              caption={Gpu Growth},
              firstline=30,
              lastline=36]
          {CodeSamples/Tensorflow/model_main_tf2.py}
    \item[2.] Falls der Trainings Prozess einen Evaluations Durchgang ausführt und der TensorflowTrainer versucht den Prozess zu schließen läuft der Pyton Prozess weiter im Hintergrund -> Schließen über TaskManager.

\end{itemize}

\subsubsection{Lite Graph}
Mit einem Klick auf \textit{Export Lite Convertible Graph} wird ein Saved Model nach ObjectDetection/export/LiteConvertibleGraph exportiert, das mit dem TFLite-Converter (\url{https://www.tensorflow.org/lite/convert}) zu einem TFLite Model convertiert werden kann.
\newpage
\subsection{Python Anwendung}
Um die trainierten Modelle verwenden zu können wurde eine Python Anwendung geschrieben, die TF 1, TF 2 und TF Lite Modelle unterstützt.
\begin{figure}[H]
    \centering
    \subfigure{\includegraphics[width=0.9\textwidth]{TensorFlow/ApplicationArchitecture.png}}
    \caption[Architektur]{Architektur}
    \label{fig:Tensorflow Architektur}
\end{figure}
Die Anwendung wurde für den Raspberry Pi 4 optimiert und verwendet alle 4 Prozessor Kerne gleichzeitg, dennoch sollte beachtet werden, dass die Rechenleistung des PI sehr beschränkt ist und nur TF Lite Modelle verwendet werden sollten.
\newpage
\subsubsection{GUI}
Die GUI wurde sehr simpel gehalten um möglichst wenig CPU-Time zu beanspruchen.
\begin{figure}[H]
    \centering
    \subfigure{\includegraphics[width=1\textwidth]{TensorFlow/GUI.PNG}}
    \caption[Pyton Gui]{Pyton Gui}
    \label{fig:Tensorflow GUI}
\end{figure}
Mit den \textit{Stop} und  \textit{Restart} Buttons können die einzelnen Threads des Programms angehalten oder neu gestartet werden. In der Linken oberen Ecke sieht man die derzeitige CPU Auslastung und die letzte Tensorflow-Detection Zeit. Wenn man den Tensorflow Thread angehalten hat ist es Möglich den Kran direkt über die Richtungs Buttons zu steuern.

\subsubsection{Einstellungen}
Die Einstellungen können über Variablen in \textit{CraneControl.py} angepasst werden.

\lstinputlisting[language=python,label={lst:CraneControl Settings},
    numbers={none},
    caption={CraneControl Settings},
    firstline=24,
    lastline=40]
{CodeSamples/Tensorflow/CraneControl.py}


\subsubsection{Setup und Ausführung}
Die Anwendung wird durch Aufrufen von \textit{CraneControl.py} mit Python gestartet.
Benötiget pip Packete: Pillow,opencv-python,psutil,tensorflow (nur x86/x64)

\textbf{Für PI:} \\
Um die Anwendung auf einem PI verwenden zu können muss der TensorFlow Lite-Interpreter installiert werden(\url{https://www.tensorflow.org/lite/guide/python}).
Außerdem müssen einige APT Packete installiert werden. (python3-pil python3-pil.imagetk libatlas-base-dev libhdf5-dev libhdf5-serial-dev libatlas-base-dev libjasper-dev libqtgui4 libqt4-test).
Auf dem Pi können nur Lite-Graphen ausgeführt werden.


\textbf{Mitgelieferte Modelle:}
\begin{itemize}
    \item mobilnet_300x300v2.tflite [TFLite] (Ungenau aber sehr schnell)
    \item mobilnet_300x300.pb [TF1] (Genau aber schnell)
    \item ResNet 50 [TF2] (Sehr genau aber langsam)
\end{itemize}

\chapter{Vergleich}
\section{Vorteile}
\textbf{Kinect}
\begin{itemize}
    \item Einfache Implementierung, da ein Großteil der Datenverarbeitung direkt in der Kinect stattfindet
\end{itemize}
\textbf{Tensorflow}
\begin{itemize}
    \item Keine spezielle Kamera notwendig.
    \item Theoretisch höhere Genauigkeit.
    \item Kontrolle über den gesamten Prozess. (Kamera bis Model Output) 
\end{itemize}
\section{Nachteile}
\textbf{Kinect}
\begin{itemize}
    \item Es ist immer eine Kinect notwendig
\end{itemize}
\textbf{Tensorflow}
\begin{itemize}
    \item Aufwendiges trainierten eines TF-Modells
    \item Hohe Genauigkeiten nur mit Großen Modellen möglich (Exponentieller Rechenleistungs zuwachs)
    \item Für Echtzeit-Erkennung wird eine GPU oder TPU benötigt.
\end{itemize}
\section{Fazit}
\chapter{Zusammenfassung}


\section{Stand bei Abgabe}
Abgegeben werden 2 fertige Projekte, die alle Anforderungen erfüllen. Zusätslich ist es einen Dummy Client zum Testen der Anwendunghen,
und einen Model-Trainer für die Tensorflow Anwendung enthalten.\\
Beide Anwendung sind voll funktionsfähig und können einfach, auch durch dritte, eingesetzt werden.


\section{Ausblick}
Durch die Neuerungen und Verbesserungen läuft die Krahnsteuerung nun zuverlässiger und ist einfacher zu bedienen. Dadurch eignet sich diese Version
besser als Vorzeigeobjekt für z.B. den Tag der offenen Tür.

% Literaturverzeichnis
\phantomsection
\addcontentsline{toc}{chapter}{Literaturverzeichnis}
\bibliographystyle{natdin}
\bibliography{literatur}
\newpage

% Abbildungsverzeichnis
\phantomsection
\addcontentsline{toc}{chapter}{Abbildungsverzeichnis}
\listoffigures
\newpage

%Tabellenverzeichnis
%\phantomsection
%\addcontentsline{toc}{chapter}{Tabellenverzeichnis}
%\listoftables
%\newpage

%Listingsverzeichnis
\phantomsection
\addcontentsline{toc}{chapter}{Listingsverzeichnis}
\lstlistoflistings
\newpage

% Anhang
%\include{anhang}
\end{document}