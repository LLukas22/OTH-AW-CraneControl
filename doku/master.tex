\documentclass[12pt,oneside]{report}
\usepackage[T1]{fontenc}		% Einstellungen fuer Umlaute usw.
\usepackage[utf8x]{inputenc}
\usepackage[ngerman]{babel}

\usepackage{parskip}			% Einstellungen fuer Absaetze: Abstand statt Einrueckung

\usepackage[a4paper,			% Papierformat A4
	    left=2.5cm,				% linker Rand
	    right=2.5cm,			% rechter Rand
	    top=1.5cm,				% oberer Rand
	    bottom=1.5cm,			% unter Rand
	    marginparsep=5mm,		% Abstand der Randnotizen
	    marginparwidth=10mm, 	% Breite der Randnotizen
	    headheight=7mm,			% Hoehe der Kopfzeile
	    headsep=1.2cm,			% Abstand der Kopfzeile
	    footskip=1.5cm,			% Abstand der Fusszeile
	    includeheadfoot]{geometry}

        \usepackage{fancyhdr}						% Konfiguration von Kopf- und Fusszeilen
        \pagestyle{fancy}							% Seitenstil 'fancy'
        \fancyhf{}									% vorhandene Einstellungen loeschen
        \setlength{\headwidth}{\textwidth}			% Kopf- und Fusszeile so breit wie der Haupttext
        \fancyfoot[R]{\thepage} 					% Festlegung des Seitenstils: Seitenzahlen in der Fusszeile rechts
        \fancyfoot[L]{\leftmark}					% Kapitelnr. und -Bezeichnung in der Fusszeile links
        \fancyhead[R]{\IhreArbeit}					% "Bachelorarbeit" in der Kopfzeile rechts
        \fancyhead[L]{\ErsterName\ , \ZweiterName}	% Vorname und Name in der Kopfzeile links
        \renewcommand{\chaptermark}[1]{			% Definition der Ausgabe des Kapitels
          \markboth{Kapitel \thechapter. #1}{}}
        \renewcommand{\headrulewidth}{0.5pt}		% Trennlinie zwischen Kopfzeile und Haupttext
        \renewcommand{\footrulewidth}{0.5pt}		% Trennlinie zwischen Haupttext und Fusszeile
        \fancypagestyle{plain}{					% Anpassung des Seitenstils 'plain' bei Beginn neuer Kapitel
          \fancyhf{}								% Vorbelegung loeschen
          \fancyfoot[C]{\thepage}					% Seitenzeilen in der Fusszeile mittig
          \fancyhead[R]{\IhreArbeit}				% "Bachelorarbeit" in der Kopfzeile rechts
          \fancyhead[L]{\ErsterName\ , \ZweiterName}	% Vorname und Name in der Kopfzeile links
        }
        
        %\usepackage{amsmath}			% Pakete fuer den Mathematikmodus
        %\usepackage{amssymb}
        \usepackage[intlimits]{empheq}
        \usepackage{subfigure}
        \usepackage{float}
        
        \usepackage[sc]{mathpazo}		% Schriftart Palatino fuer Haupttext und Mathematikmodus
        \usepackage{pifont}				% zusaetzliche Symbole
        
        \usepackage[format=hang,		% Einstellung fuer Bildunterschriften
                    font={footnotesize},
                    labelfont={bf},
                    margin=1cm,
                    aboveskip=5pt,
                    position=bottom]{caption}
        
        \usepackage{graphicx}	   % Einbinden von Grafiken (jpg, png, pdf, ...)
        \graphicspath{{images/}}   % Suchpfad für Grafikdateien
        
        \usepackage[svgnames,table,hyperref]{xcolor} 	% Verwendung von Farben
        \usepackage{tikz}								% Erstellen von Grafiken
        \usetikzlibrary{positioning,arrows,plotmarks} % TikZ-Bibliotheken
        %\usepackage{pgfplots}                           % Darstellung von Plots, Funktionen, Graphen usw.
        
        %
        % Weitere Pakete
        %
        \usepackage{listings}			% Darstellung von Quellcode
        \lstset{language=Python, basicstyle=\ttfamily, numbers=none}
        %
        \usepackage[square,numbers,sort]{natbib} % Referenzen
        %
        %\usepackage[european, siunitx]{circuitikz}	% Darstellung von Schaltungen
        %
        %\usepackage{enumerate}			% Formatierung nummerierter Listen
        
        \usepackage{microtype,relsize}					% Wird verwendet, um Nachnamen auf Titelseite gesperrt darzustellen
        \newcommand*{\Sperren}[1]{\textls*[100]{#1}}

        
% 
% Persoenliche Angaben
% 
\newcommand*{\ErsterName}{Lukas Kreussel}
\newcommand*{\ZweiterName}{Andreas Ziegler}
\newcommand*{\IhrStudiengang}{Elektro- und Informationstechnik}
\newcommand*{\IhreArbeit}{Projektarbeit}
\newcommand*{\IhrTitelDE}{Verbesserung der Gestensteuerung auf Basis eines Kinect 2.0 Moduls}
\newcommand*{\Professor}{Prof. Dr.-Ing. Armin Wolfram}
\newcommand*{\IhreSchluesselwoerter}{Kinect, Tensorflow, Gestensteuerung}


\usepackage[bookmarks, raiselinks, pageanchor, % PDF-Einstellungen
            hyperindex, colorlinks,
            citecolor=black, linkcolor=black,
            urlcolor=black, filecolor=black,
            menucolor=black]{hyperref}
\hypersetup{pdftitle={\IhrTitelDE},%
            pdfauthor={\ErsterName\ \ZweiterName},%
            pdfsubject={\IhreArbeit},%
            pdfkeywords={\IhreSchluesselwoerter}}



\begin{document}
  \pagenumbering{roman}

  \thispagestyle{empty}			% 2. Seite wie Titelseite, aber mit zusaetzlichen Angaben
  \begin{center}
    \Large
    Ostbayerische Technische Hochschule Amberg-Weiden\\
    Fakultät Elektrotechnik, Medien und Informatik\\[1cm]
    Studiengang \IhrStudiengang\\[1cm]
    \textbf{\IhreArbeit}\\[1cm]
    von\\[1cm]
    \ErsterName\ , \ZweiterName\\[1cm]
    \textbf{\IhrTitelDE}\\[1cm]
  \end{center}
  \vspace*{5cm}

  \vspace*{1cm}
  \underbar{Professor:}\qquad\Professor
  \clearpage
  \tableofcontents
  \newpage
  
  
  \pagenumbering{arabic}
  \chapter{Einleitung}


\section{Zielsetzung}
Mit der Kinect 2.0 oder eine Raspberry Pi mit Kamera sollen 2 Möglichkeiten zur Verbesserung der Gestensteuerung des Krans im Labor,
Abbildung 1.1, implementiert werden.
\begin{figure}[H]
	\centering
	\includegraphics[width=0.9\linewidth]{Krahn.jpg}
	\caption[Krahn]{Krahn. Bildquelle: Eigenes Bild}
	\label{fig:Krahn}
\end{figure}

\newpage
\section{Kinect}
\begin{figure}[H]
	\centering
	\includegraphics[width=0.7\linewidth]{Xbox-One-Kinect.jpg}
	\caption[Kinect v2]{Kinect v2. Bildquelle: Evan-Amos / Wikipedia public domain}
	\label{fig:pi4}
\end{figure}

Die Kinect 2 von Microsoft ist der Nachfolger der Ursprüglichen Kinect für die xBox 360 und wurde vorallem für Spiele entwockelt,
fand jedoch in diesem Gebiet wenig Markterfolg. Darum wurde die Kinect für xBox nach der Kinect 2 eingestellt. Seit dem findet die Kinect vor allem
Anwendung in Kameravison Anwendungen. Beispielsweise ist hier das Ganzkörpertracking für VR-Anwendungen zu nennen. Durch diese Entwicklung hat sich
Microsoft dazu entschieden, die Kinect Azure für Unternehmen zu entwickeln und als Entwicklungsumgebung zu vermarkten.\\
Durch die 3 Kameras, wovon eine Infrarot- und eine Tiefenkamera verbaut sind, und den on-Board Prozessor kann die Kinect in Anwendungen viel Arbeit
abnehmen, da sie bereits viele Daten auswertet und verarbeitet. So giebt die Kinect dem Anwender bis zu 6 Ganzkörper Skelette mit einzelnen "Gelenken"
die danach weiter verarbeitet werden können. Außerdem ist einen Gesten und Mimikerkennung bereits mit verbaut.\\
Für die Entwicklung von Kinectanwendungen stellt Microsoft ein umfangreiches Software Developer Kit SDK 2.0 \cite{KinectSDK} zu Verfügung. Darüber kann
ein Kinectobject erzeugt werden, dass alle Sensorausgaben und Skelettdaten als Membervariablen ausgiebt.

\newpage
\section{Tensorflow}
\begin{figure}[H]
	\centering
	\includegraphics[width=0.7\linewidth]{TensorFlow/tf_logo.png}
	\label{fig:TF Logo}
\end{figure}
TensorFlow ist ein, von Google entwickeltes, Framework zur datenstromorientierten Programmierung.
Populäre Anwendung findet TensorFlow im Bereich des maschinellen Lernens.
Seit 2020 wurde die TensorFlow Object Detection API offiziel in Tensorflow 2.0 integriert und ermöglicht das einfache und schnelle erstellen von ObjectDetection Modellen.
Hierfür werden von Google mehrere vortränierte Modelle über den Detection Model Zoo bereitgestellt, die innerhalb von wenigen Stunden und mit relativ kleinen Datenmengen auf einen neuen Datensatz gangepasst werden können.

\newpage
\section{Anforderungen}
Der Krahn soll sich über Wlan an einem Server, der lokal auf einem Rechner im Labor läuft, anmelden und dieser soll die Daten aus der Erkennung von
Kinect oder TensorFlow an den Krahn auf Anfrage übermitteln.\\
\\\
Für die Bewegung des Krahns sollen Gesten implementiert werden:
\begin{itemize}
	\item Bewegeung nach links und rechts des Krahns
	\item Einziehen und Ausfahren des Krahnseils
	\item Ein- und Ausschalten des Reglers für eine stabile Last
	\item Umschalten, zwischen den Betriebsmodi
\end{itemize}
  \chapter{Herangehensweise}
Durch die Anforderung von 2 Möglichkeiten wurden diese Aufgeteilt und jeder hat sich mit einer Variante beschäftigt.

\section{Kinect}
Zuerst wurde das Kinect 2.0 SDK heruntergeladen und installiert. Danach wurde ein Beispielproject gesucht \cite{KinectFingerTracking}, welches sich
auch zum Teil im Fertigen Project wiederfindet. Es dient der Veranschaulichung, welche Hände momentan erkannt werden,
und zeichnet um diese einen Rahmen herum. Danach wurde auf Basis der Vordefinierten Gesten der Kinect Gesten für die Krahnbewegungen festgelegt und
die Geste für Bewgung mit einer Richtungsangabe über das Skelett versehen. Diese wurden dann an den Server gegebn, das dieser diese dem Krahn auf
Anfrage mitteilen kann.
\section{Tensorflow}
  \chapter{Übertragung}
Das Programm öffnet einem Server mit dem es möglich ist die Daten per TCP/IP abzugreifen. Die Daten werden in ein 5 Byte großes Array geschrieben. Dies wird dann an den Client gesendet. Im Moment läuft der Server auf dem Port 54000. Es liegt ein Beispielprogramm eines Clients bei, welcher die Daten abgreift.


\begin{table}[H]
    \centering
    \begin{tabular}{r|c|c|c|c|c}
        Byteposition	&0&	1&	2&	3&	4\\\hline
        Bedeutung&	Power &Ein/Aus&	Links&	Rechts&	Auf	Ab\\\hline
        Werte	&0,1&	0…100&	0.100	&0,1,2&	0,1,2
    \end{tabular}
    \caption{Übertragungsprotokoll}
    \label{tab:Protokoll}
\end{table}


  \chapter{Implementierung}


\section{Kinect - C\#}
\subsection{Konzept}
Kinect erkennt 3 Gesten + NotTracked + Unknown

\begin{figure}[H]
    \centering
    \subfigure[Offen]{\includegraphics[width=0.32\textwidth]{Geste_offen.jpg}}
    \subfigure[Geschlossen]{\includegraphics[width=0.32\textwidth]{Geste_geschlossen.jpg}}
    \subfigure[Lasso]{\includegraphics[width=0.32\textwidth]{Geste_lasso.jpg}}
    \caption[Kinect Gesten]{Kinect Gesten. Bildquelle: eigene Bilder}
	\label{fig:Kinect Gesten}
\end{figure}


\subsection{Code}
\subsubsection{Gestenerkennung}
\begin{figure}[H]
    \centering
    \begin{tikzpicture}
        \newcommand{\xa}{5};
        \newcommand{\xb}{3};
        \draw[draw=lightgray, step=1cm] (-{\xa}, -{\xa}) grid ({\xa}, {\xa});
        \draw[thick, -latex] (-{\xa+0.5}, 0) -- ({\xa-0.5}, 0) node[below] {$x$};
        \draw[thick, -latex] (0, -{\xa+0.5}) -- (0, {\xa-0.5}) node[right] {$y$};
        \node [draw] at (0,{\xa+0.5}) {UP};
        \node [draw] at (0,-{\xa-0.5}) {DOWN};
        \node [draw] at (-{\xa-1},0) {LEFT};
        \node [draw] at ({\xa+1},0) {RIGHT};
        \draw [draw=red, fill=red] (0,0) -- ({\xb},{\xa}) -- ({\xa},{\xa}) -- ({\xa},{\xb}) -- cycle;
        \draw [draw=red, fill=red] (0,0) -- (-{\xb},-{\xa}) -- (-{\xa},-{\xa}) -- (-{\xa},-{\xb}) -- cycle;
        \draw [draw=red, fill=red] (0,0) -- ({\xb},-{\xa}) -- ({\xa},-{\xa}) -- ({\xa},-{\xb}) -- cycle;
        \draw [draw=red, fill=red] (0,0) -- (-{\xb},{\xa}) -- (-{\xa},{\xa}) -- (-{\xa},{\xb}) -- cycle;
    \end{tikzpicture}
    \caption{Filter der Richtungsgesten.}
    \label{fig:Richtungsfilter}
\end{figure}
\subsubsection{Kinect}
Übernimmt die Initialiserung und Verabeitung der Kinect. Ausgegeben werden eine Bitmap des Bildes und in einem Array die Richtungsanweisungen.
Im Bild werden die erkannten Hände umrandet.

\subsubsection{Krahnsteuerung}
Ist das Hauptprogramm in dem das Winform und der Server gehandhabt werden.

\subsubsection{SharedRessources}
Klasse, in der die Übergabe an das Hauptprogramm geregelt wird, so dass dieses um andere Eingabemöglichkeiten erweitert werden könnte.

\subsection{UI}
UI
\begin{figure}[H]
    \centering
        \includegraphics[width=0.7\linewidth]{Kinect UI.png}
        \caption[Kinect UI]{Kinect UI}
        \label{fig:Kinect UI}
\end{figure}




\newpage
\section{Tensorflow - Python}
\subsection{Konzept}
Es wird ein Object-Detection-Model verwendet um über eine Webcam eine von sechs Gesten zu erkennen.
Die erkannte Geste wird in einem Buffer zwischengespeichert, und wenn mehrere Durchläufe das gleiche Ergebniss liefern wird ein Befehl an den Kran gesendet.

\begin{figure}[H]
    \centering
    \subfigure[Power]{\includegraphics[width=0.3\textwidth]{TensorFlow/Gestures/Power.PNG}}
    \subfigure[Up]{\includegraphics[width=0.3\textwidth]{TensorFlow/Gestures/Up.PNG}}
    \subfigure[Down]{\includegraphics[width=0.3\textwidth]{TensorFlow/Gestures/Down.PNG}}
    \subfigure[Left]{\includegraphics[width=0.3\textwidth]{TensorFlow/Gestures/Left.PNG}}
    \subfigure[Right]{\includegraphics[width=0.3\textwidth]{TensorFlow/Gestures/Right.PNG}}
    \subfigure[Toggle]{\includegraphics[width=0.3\textwidth]{TensorFlow/Gestures/Toggle.PNG}}
    \caption[Tensorflow Gesten]{Tensorflow Gesten. Bildquelle: eigene Bilder}
	\label{fig:Tensorflow Gesten}
\end{figure}
\newpage
\subsection{Tensorflow Trainer}
Um das tränieren der Modele zu erleichtern wurde eine C\# Application geschrieben, die diesen Vorgang weitgehend automatisiert.

\begin{figure}[H]
    \centering
    \subfigure{\includegraphics[width=0.9\textwidth]{TensorFlow/TensorflowTrainer.PNG}}
    \caption[Tensorflow Trainer]{Tensorflow Trainer}
	\label{fig:Tensorflow Trainer}
\end{figure}

\subsection{Tränieren eines eigene Models}

\subsubsection{Voraussetzungen:}

% \begin{tensorflowTrainerPrerequisites}[H]
%     \item Git (\url{https://git-scm.com/downloads})
%     \item C++ Build Tools (\url{https://visualstudio.microsoft.com/thank-you-downloading-visual-studio/?sku=BuildTools&rel=16})
%     \item Python 3.7.X or newer (\url{https://www.python.org/downloads/})
%     \item LabelImg (\url{https://github.com/tzutalin/labelImg})
%     \item Nvidia Gpu (optional but highly recommended)
%     \item Download Matching CUDA Version (\url{https://www.tensorflow.org/install/gpu})
%     \item Matching cuDNN Version (\url{https://developer.nvidia.com/cudnn})
% \end{tensorflowTrainerPrerequisites}




\textbf{TensorflowTrainer:} \\
    Git: \url{https://git-scm.com/downloads}\\
    Suppoerted Models: \url{https://github.com/tensorflow/models/blob/master/research/object_detection/g3doc/tf2_detection_zoo.md}\\
    Microsoft C++ Build Tools: \url{https://visualstudio.microsoft.com/thank-you-downloading-visual-studio/?sku=BuildTools&rel=16}\\
    Für GPU-Support: \url{https://www.tensorflow.org/install/gpu}\\


\textbf{Für PI:} \\
    TFLight : \url{https://www.tensorflow.org/lite/guide/python}\\
    Apt.Get : sudo apt-get install python3-pil python3-pil.imagetk libatlas-base-dev libhdf5-dev libhdf5-serial-dev libatlas-base-dev libjasper-dev libqtgui4 libqt4-test\\
    Pip3: pip3 install Pillow,opencv-python\\



  \chapter{Vergleich}
\section{Vorteile}
\textbf{Kinect}
\begin{itemize}
    \item Einfache Implementierung, da ein Großteil der Datenverarbeitung direkt in der Kinect stattfindet
\end{itemize}
\textbf{Tensorflow}
\begin{itemize}
    \item Keine spezielle Kamera notwendig.
    \item Theoretisch höhere Genauigkeit.
    \item Kontrolle über den gesamten Prozess. (Kamera bis Model Output) 
\end{itemize}
\section{Nachteile}
\textbf{Kinect}
\begin{itemize}
    \item Es ist immer eine Kinect notwendig
\end{itemize}
\textbf{Tensorflow}
\begin{itemize}
    \item Aufwendiges trainierten eines TF-Modells
    \item Hohe Genauigkeiten nur mit Großen Modellen möglich (Exponentieller Rechenleistungs zuwachs)
    \item Für Echtzeit-Erkennung wird eine GPU oder TPU benötigt.
\end{itemize}
\section{Fazit}
  \chapter{Zusammenfassung}


\section{Stand bei Abgabe}
Abgegeben werden 2 fertige Projekte, die alle Anforderungen erfüllen. Zusätslich ist es einen Dummy Client zum Testen der Anwendunghen,
und einen Model-Trainer für die Tensorflow Anwendung enthalten.\\
Beide Anwendung sind voll funktionsfähig und können einfach, auch durch dritte, eingesetzt werden.


\section{Ausblick}
Durch die Neuerungen und Verbesserungen läuft die Krahnsteuerung nun zuverlässiger und ist einfacher zu bedienen. Dadurch eignet sich diese Version
besser als Vorzeigeobjekt für z.B. den Tag der offenen Tür.

  
    
  % Abbildungsverzeichnis
  \phantomsection
  \addcontentsline{toc}{chapter}{Abbildungsverzeichnis}
  \listoffigures
  \newpage

  %Tabellenverzeichnis
  \phantomsection
  \addcontentsline{toc}{chapter}{Tabellenverzeichnis}
  \listoftables
  \newpage

  % Anhang
  %\include{anhang}
\end{document}    